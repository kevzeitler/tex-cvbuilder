\documentclass[10pt, a4paper]{article}

\input{cv-themes/theme_selection.tex}

%----------------------------------------------------------------------------------------
%	 THEMES
%----------------------------------------------------------------------------------------

% Define the desired theme out of the following: beige, blue, bw, coral, earth, framed, gray, minimal, onyx, plain
% See screenshots in preview/ directory
\theme{bw}

%----------------------------------------------------------------------------------------
%	 PERSONAL INFORMATION
%----------------------------------------------------------------------------------------

% If you don't need a particular field, just remove the content leaving the command, e.g. \aboutdesc{}

\name{Kevin Zeitler} % Your name
\jobtitle{Lebenslauf} % Job title/career
\profilepic{kevinzeitler.png} % Profile picture (supported only with "bw", "gray" and "framed" themes)
%\aboutdesc{Coburg} % Description for ABOUT ME section
\aboutdesc{Softwareingenieur mit dem Schwerpunkt auf eingebetteten Systemen und Linux im Automobilbereich. Meine Neugier und Leidenschaft für die Luftfahrt treiben mich an, meine Fähigkeiten in neuen Branchen einzusetzen.}
\location{Deggendorf, Deutschland} % Address/location
\phone{+49 1515 5578674} % Phone number
\mail{kevin.zeitler@mailbox.org} % Mail
\dateofbirth{30 April 1999} % Date of birth
\drivinglicenses{Klasse B} % Drivers license category
\linkedin{\href{https://de.linkedin.com/in/kevin-zeitler}{linkedin.com/in/kevin-zeitler}} % \linkedin{\href{LINK}{DESCRIPTION}}
\github{} % \github{\href{{LINK}{DESCRIPTION}}

%----------------------------------------------------------------------------------------
%	 SKILLS
%----------------------------------------------------------------------------------------

% Both proskills and perskills can be used separately or together. If you do not plan to use professional and personal skill charts, just remove the content leaving the command, e.g. \perskills{} or \proskills{}

% Define professional skills (values are from interval [0,1])
%\proskills{{C++/0.9},{Java/0.7},{Python/0.45}}
\proskills{}

% Define personal skills (values are from interval [0,1])
%\perskills{{Zuverlässigkeit/0.9},{Belastbarkeit/0.7},{Teamfähigkeit/1.0}}
\perskills{}

% Define language skills (values are from interval [0,5). If you do not plan to use the language skills chart, just remove the content leaving the command, e.g. \langskills{}
% Language skill circles are designed to be included into the sidebar
\langskills{{Deutsch/5},{Englisch/4},{Spanisch/2},{Russisch/2},{Französisch/1}}
%\langskills{}

\begin{document}

\makeprofile % Print name & job description. Also prints out profile picture if it's supported by the theme

\begincols

%----------------------------------------------------------------------------------------
%	 SIDEBAR
%----------------------------------------------------------------------------------------
% Use \subsection inside the sidebar

% Print defined contact information
% \makecontact{NAME FOR CONTACT SECTION}
\makecontact{KONTAKT}

% \makedob{NAME FOR DATE OF BIRTH SECTION}
\makedob{GEBURTSTAG}

% \makelicense{NAME FOR ABOUT ME SECTION}
\makeabout{ÜBER MICH}

% \makelicense{NAME FOR DRIVING LICENSE SECTION}
\makelicense{FÜHRERSCHEIN}

\subsection{SPRACHEN}
\langcircles % Command for drawing language skill circles

%\subsection{STÄRKEN}
%\perskills % Command for drawing personal skill bars.

%\subsection{PROFESSIONAL}
%\proskills % Command for drawing professional skill bars.

%----------------------------------------------------------------------------------------
%	 BODY
%----------------------------------------------------------------------------------------
% Use \section inside the body
\switchcols % This command is used to switch to the document body

\section{BERUFSERFAHRUNG}

% \begin{cvitem}{Job title}{Company name}{Location}{Duration}
% Job description
% \end{cvitem}
% To add bullets inside the job description:
% \begin{cvitem}{Job title}{Company name}{Location}{Duration}
%   \begin{itemize}
%       \item
%   \end{itemize}
% \end{cvitem}

\begin{cvitem}{Softwareingenieur}{M-Sys GmbH}{Wallersdorf}{SEIT 2021}
%Lorem ipsum dolor sit amet, consectetur adipiscing elit, sed do eiusmod tempor incididunt ut labore et dolore magna aliqua.
%    \begin{itemize}
%        \item Lorem ipsum dolor sit amet
%    \end{itemize}
\end{cvitem}

\section{PRAKTIKA}

\begin{cvitem}{Softwareentwickler}{M-Sys GmbH}{Wallersdorf}{2020}
    Praxissemester
\end{cvitem}

\begin{cvitem}{Softwareentwickler}{M-Sys GmbH}{Wallersdorf}{2019-2020}
    Freiwilliges Praktikum
\end{cvitem}

\begin{cvitem}{}{Lufthansa Technical Training GmbH}{Frankfurt}{2015}
    Freiwilliges Praktikum
\end{cvitem}

\section{BILDUNG}

% \begin{cvitem}{Education level}{College/High school}{Location}{Duration}
% Description
% \end{cvitem}

\begin{cvitem}{Bachelor Sc.}{Hochschule für angewandte Wissenschaften}{Landshut}{2017-2021}
    Studium der Automobilinformatik
\end{cvitem}

\begin{cvitem}{Abitur}{Hennebergisches Gymnasium "Georg Ernst"}{Schleusingen}{2009-2017}
    Allgemeine Hochschulreife
\end{cvitem}

%\section{SPRACHEN}

%\begin{langs}
%   \langsitem{Language}{Native} % Optionally add \cline{2-3} after a row to insert a semi-line between languages
%    \langsitem{English}{Native} \cline{2-3}
%   \langsitemmulti{Language}{Understanding level}{Speaking level}{Writing level}
%    \langsitemmulti{German}{B1}{B2}{B1}
%\end{langs}

%\cefrdesc % Use this to add CEFR description, otherwise delete or comment it out

\section{FREIZEIT}

\begin{projitem}{Sport}{}{}
    \begin{itemize}
        \item Laufen
        \item Wandern
    \end{itemize}
\end{projitem}

\begin{projitem}{Ehrenamt}{Campus Landshut e.V.}{2018-2021}
   \begin{itemize}
        \item Erweitertes Vorstandsmitglied 
        \item PR Abteilungsmitglied
    \end{itemize}
\end{projitem}

%\section{PROJECTS}

% \begin{projitem}{Project name}{\href{link}{link description}}
% Project description
% \end{projitem}

%\begin{projitem}{Project name}{\href{https://link}{project link}}
%    Lorem ipsum dolor sit amet, consectetur adipiscing elit, sed do eiusmod tempor incididunt ut labore et dolore magna aliqua.
%\end{projitem}

% Use \skillsbars to draw professional skill bars next to the personal skill bars
% \skillsbars{NAME FOR PROFESSIONAL SKILLS SECTION}{NAME FOR PROFESSIONAL SKILLS SECTION}
\skillsbars{PROFESSIONAL}{PERSONAL}

\fincols
\end{document}
